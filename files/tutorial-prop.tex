\documentclass[]{article}
\usepackage{lmodern}
\usepackage{amssymb,amsmath}
\usepackage{ifxetex,ifluatex}
\usepackage{fixltx2e} % provides \textsubscript
\ifnum 0\ifxetex 1\fi\ifluatex 1\fi=0 % if pdftex
  \usepackage[T1]{fontenc}
  \usepackage[utf8]{inputenc}
\else % if luatex or xelatex
  \ifxetex
    \usepackage{mathspec}
    \usepackage{xltxtra,xunicode}
  \else
    \usepackage{fontspec}
  \fi
  \defaultfontfeatures{Mapping=tex-text,Scale=MatchLowercase}
  \newcommand{\euro}{€}
\fi
% use upquote if available, for straight quotes in verbatim environments
\IfFileExists{upquote.sty}{\usepackage{upquote}}{}
% use microtype if available
\IfFileExists{microtype.sty}{%
\usepackage{microtype}
\UseMicrotypeSet[protrusion]{basicmath} % disable protrusion for tt fonts
}{}
\usepackage[margin=1in]{geometry}
\ifxetex
  \usepackage[setpagesize=false, % page size defined by xetex
              unicode=false, % unicode breaks when used with xetex
              xetex]{hyperref}
\else
  \usepackage[unicode=true]{hyperref}
\fi
\hypersetup{breaklinks=true,
            bookmarks=true,
            pdfauthor={},
            pdftitle={useR! 2016 Tutorial Proposal},
            colorlinks=true,
            citecolor=blue,
            urlcolor=blue,
            linkcolor=magenta,
            pdfborder={0 0 0}}
\urlstyle{same}  % don't use monospace font for urls
\usepackage{longtable,booktabs}
\setlength{\parindent}{0pt}
\setlength{\parskip}{6pt plus 2pt minus 1pt}
\setlength{\emergencystretch}{3em}  % prevent overfull lines
\providecommand{\tightlist}{%
  \setlength{\itemsep}{0pt}\setlength{\parskip}{0pt}}
\setcounter{secnumdepth}{0}

%%% Use protect on footnotes to avoid problems with footnotes in titles
\let\rmarkdownfootnote\footnote%
\def\footnote{\protect\rmarkdownfootnote}

%%% Change title format to be more compact
\usepackage{titling}

% Create subtitle command for use in maketitle
\newcommand{\subtitle}[1]{
  \posttitle{
    \begin{center}\large#1\end{center}
    }
}

\setlength{\droptitle}{-2em}
  \title{useR! 2016 Tutorial Proposal}
  \pretitle{\vspace{\droptitle}\centering\huge}
  \posttitle{\par}
  \author{}
  \preauthor{}\postauthor{}
  \date{}
  \predate{}\postdate{}


% Redefines (sub)paragraphs to behave more like sections
\ifx\paragraph\undefined\else
\let\oldparagraph\paragraph
\renewcommand{\paragraph}[1]{\oldparagraph{#1}\mbox{}}
\fi
\ifx\subparagraph\undefined\else
\let\oldsubparagraph\subparagraph
\renewcommand{\subparagraph}[1]{\oldsubparagraph{#1}\mbox{}}
\fi

\begin{document}
\maketitle

\emph{Please fill out the sections. In your final document be sure to
remove instructive comments such as this one.}

\subsection{Tutorial Title}\label{tutorial-title}

\emph{The title for the tutorial.}

\subsection{Instructor Details}\label{instructor-details}

\begin{longtable}[c]{@{}ll@{}}
\toprule
Name: &\tabularnewline
&\tabularnewline
Institution: &\tabularnewline
&\tabularnewline
Address: &\tabularnewline
&\tabularnewline
Email: &\tabularnewline
\bottomrule
\end{longtable}

\subsection{Short Instructor
Biography}\label{short-instructor-biography}

\emph{The bio should enable the scientific committee to gauge the
instructor's qualifications for the tutorial. Links to works previous
tutorials and/or relevant publications may be included.}

\subsection{Brief Description of
Tutorial}\label{brief-description-of-tutorial}

\emph{Describe the tutorial briefly.}

\subsection{Goals}\label{goals}

\begin{enumerate}
\def\labelenumi{\arabic{enumi}.}
\tightlist
\item
  \emph{First goal}
\item
  \emph{Second goal}
\item
  \emph{Etc., numbered as shown}
\end{enumerate}

\subsection{Detailed Outline}\label{detailed-outline}

\emph{Detailed outline text.}

\subsection{Justification}\label{justification}

\emph{Justification why tutorial is current and important.}

\subsection{Background Knowledge}\label{background-knowledge}

\emph{Background knowledge required of attendees.}

\subsection{Expected Number of
Attendees}\label{expected-number-of-attendees}

\emph{How many do you expect to attend the tutorial?}

\end{document}
